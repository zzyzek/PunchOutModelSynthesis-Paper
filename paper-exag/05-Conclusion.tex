\section{Conclusion}

Punch Out Model Synthesis (POMS) provides an algorithm for the Constraint Based Tiling Generation (CBTG) problem.
We have shown that POMS can discover realizations from tile constraints that have finite correlation length.
We have also shown that POMS is able to find realizations for some problems
that have weak implied global constraints.
If the tile set and configuration are not conditioned well,
POMS may fail to find a solution or provide biased results.

Tile constraints that have unbounded or long range correlations are more difficult and sometimes intractable.
Constraint Based Tiling Generation problems problems are, in general, NP-Complete, so there is likely
no comprehensive strategy that leads to efficient methods of solution but the worst case complexity
results sometimes obscure when problems are readily solvable.
For many CBTG problems that are NP-Complete, the general complexity result might not
apply to some generic configuration, allowing some problem ensembles
to be easily solvable.

For many real world CBTG problems, we still lack understanding of the
interplay between how difficult it is to find realizations given tile constraints
and initial configuration.
The Tile Arc Consistent Correlation Length (TACCL) is one heuristic measure that attempts
to quantify how difficult tile constraints are to resolve.
The TACCL has the benefit of being easy to calculate but its interpretation is easily confounded,
so should be considered a coarse measure with limited applicability.

A libre/free reference implementation for Punch Out Model Synthesis (POMS) has been developed and can be downloaded
from its repository \textsuperscript{ \ref{poms-url} }.
%from its repository FOOTNOTE_REPO_IDENT.tex.

