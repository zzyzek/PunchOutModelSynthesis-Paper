
%\appendix

\section{Appendix: Breakout Model Synthesis (BMS)}

% the 'H' below forces the position so it doesn't float
% to the left of the references (as it keeps doing for
% some reason)
%

%\raggedleft\begin{algorithm}
\begin{algorithm}[H]
  \caption{Breakout Model Synthesis}
  \label{alg:bms}
  \begin{algorithmic}
    \State \textbf{Input,Output:} Block $B$,
    \State \textbf{Input:} Setup restrictions $S$,
    \State \textbf{Input:} Tile constraints $C$,
    \State \textbf{Input:} Tile Choice Scheduler $TCS$
    \State \textbf{Input:} Soften Choice Scheduler $SCS$

		\State Put $B$ into a fully indeterminate state
		\State Apply setup restrictions $S$ to $B$
    \State Apply tile constraints $C$ until $B$ is arc consistent

		\If{ unable to create initial arc consistent state }
			\State return failure
		\EndIf

		\State Save copy of $B$ to $P$
    \While{ $B$ not fully resolved }
			\State $B' = B$
      \State Choose tile and cell to resolve in $B$ from $TCS$
      \State Propagate resolution until $B$ is arc consistent
      \If{ contradiction encountered }
				\State Revert $B$ back to $B'$
				\State Query $SCS$ for a sub-region, $R$, to soften
				\State Revert region $R$ in $B$ back to $P$
        \State Constraint propagate until $B$ is arc consistent
			\EndIf

      \If{ iteration count has been exceeded }
        \State return failure
      \EndIf
    \EndWhile
    \State return success
  \end{algorithmic}
\end{algorithm}



%Breakout Model Synthesis (BMS) was introduced in Hoetzlein's \texttt{just\_math} \cite{Hoetzlein_2023} project as a block level solver.
%BMS notes the existence
%of a finite arc consistent correlation length in many tile sets that can be exploited to create a stochastic backtracking
%algorithm.
%
%For tile sets that have a finite arc consistent correlation length,
%changes are likely not to impact distant regions.
%This suggests a solving strategy to recover from a contradiction
%by reverting a localized region back to an indeterminate state.
%The finite arc consistent correlation length has the potential to keep
%the reversion localized, leaving progress elsewhere unscathed.
%
%Many tile sets have implicit global constraints, which effectively act as long range arc consistent correlations.
%Though longer range correlations violate the finite arc consistent correlation length assumption,
%sometimes solvers can still succeed in finding realizations if they're able to overcome
%local level constraints.
%
%With this in mind, BMS proceeds as in WFC initially, choosing a tile to resolve based on some tile choice scheduler
%\footnote{Gumin introduced a minimum entropy heuristic as the tile choice scheduler for WFC}
%and propagating constraints.
%Should a contradiction be encountered, BMS ``softens'' a larger region by reverting it back to an indeterminate state,
%using the arc consistent correlation length to inform the soften size.
%BMS can further handle any implicit tile level constraints during initial block setup by adding or removing tiles
%and performing an initial constraint propagation.
%Randomness can be introduced in the tile choice scheduler and the soften region scheduler to help overcome simple cycles.


%For completeness, the BMS algorithm used in this paper is presented in Algorithm \ref{alg:bms}.
%Note that the soften choice scheduler ($SCS$) can return an empty region indicating,
%for example, that the soften step is done only after multiple failures have been encountered.
%Readers should refer to \ref{hoetzlein_just_math} for further details on Breakout Model Synthesis (BMS).
%
%Though BMS overcomes the limitation of WFC's one-shot nature by softening contradictory regions to stochastically backtrack, BMS
%suffers the same resource usage limitations as WFC.
%As with all stochastic solvers, BMS can get trapped in local minima if the tile set is not conditioned well, the soften region is too
%small or if other parameters are not well conditioned to the problem.
%Global constraints are varied and can cause slowdowns or failures for BMS.
%Often, there needs to be experimentation on proper parameterization for BMS's successful usage.
%
%We only note that WFC with backtracking has been proposed **REF**.
%Deterministic backtracking and stochastic exploration are two methods to explore the solution space and POMS favors
%stochastic exploration with iteration limits in favor of a complete, deterministic exploration.
%%that might come at the cost of exponential search.
%
%In this paper, BMS is used as the block level tile solver for POMS as it provides a good alternative to naive WFC.

